% !TeX root = ../thesis.tex

\chapter{总结与展望}{Conclusion and Outlook}

近年来,针对信息系统计算堆栈的底层核心硬件——
处理器及其系统的攻击开始展现出技术含量高、破坏性大、波及范围广、
难以预防及修复等特点,严重威胁着家庭、汽车、企业、网络和云端的设备。
RISC-V作为最新一代精简指令集,以其前所未有的开放性和可扩展性,
在处理器安全乃至计算机体系结构研究领域掀起了新的热潮。
目前业内领先的RISC-V处理器实现多为开源项目,
更为研究处理器内部运行机制的安全性提供了无与伦比的便利性。
本文基于RISC-V指令集生态中的两款重要硬件产品:
中科院计算所的香山高性能开源处理器,
和SiFive公司的最新硬件HiFive Unmatched单板计算机,
展示了两种隐蔽性强、容易实施、危害性大的硬件攻击实例。

本研究的工作可总结为以下几个方面:

\begin{compactenum}
	\item 调查了针对硬件攻击的研究现状,在分析各种攻击手段的基础上,
    着重分析了针对高性能处理器及其系统的攻击方案。综合考虑后,
    选择Spectre边界检查绕过攻击和DMA任意存储器地址访问攻击作为本文的研究对象。
	\item 充分研究中科院计算所的香山处理器微架构后,设计并实现了一种
    基于地址冲突替换(Evict)的高速缓存控制算法,
    用于Spectre攻击实施时缓存侧信道的建立。
	\item 基于香山处理器南湖微架构RTL仿真器验证了Spectre边界检查绕过攻击,
    实现了非授权情况下对任意存储器地址中数据的读取。
	\item 基于Xilinx KCU116 FPGA开发板,使用Vivado Block Design工具
    设计了一款用于PCIe DMA攻击的PCIe Endpoint设备。
	\item 实现了一个在Linux操作系统上,将任意进程虚拟地址空间中的虚拟地址
    翻译为物理地址的工具程序。
    \item 在HiFive Unmatched单板计算机上验证了PCIe DMA攻击,
    实现了对在Linux中任意地址的访问的攻击。
\end{compactenum}

毕业设计虽然告一段落,但由于时间所限及疫情的影响,本研究还有很大的
完善空间。在后续研究中可以进一步开展的工作有:

\begin{compactenum}
	\item 受疫情影响,香山处理器雁栖湖微架构实际流片后的开发板无法从
    北京邮寄到上海,使得本研究错失了在硬件平台上验证Spectre攻击方案的机会。
    后续可以将高速缓存Evict算法以及Spectre攻击方案移植到香山处理器硬件上验证,
    借助硬件更高的性能完善统计算法,实现更准确的存储器内容读取。
	\item 专用FPGA PCIe Endpoint的使用限制了PCIe DMA攻击的运用场景,
    但值得注意的是处理器系统中普遍安装的GPU也具有DMA传输的功能,接下来可以
    尝试通过控制GPU的DMA引擎,实现更加普遍的PCIe DMA攻击方案。
\end{compactenum}

\newpage
