% !TeX root = ../thesis.tex

\chapter{引言}{Introduction}


\section{研究背景及意义}{Background and Significance}
诞生于30年前的国际互联网使人类进入了信息大爆炸的时代,
尤其是近年出现的移动互联网让更多的人与设备接入了这个庞大的网络,
随之而来的是海量的信息以及对这些信息进行处理的需求。
人们部署了无数计算设备来处理信息,从我们身边的智能手表、
自动驾驶汽车,到云端的数据中心,
这些计算设备的安全性直接关系到我们的生活质量、隐私状况甚至人身安全。

用于处理数据的计算设备的架构中一般有两个主要部件:硬件基础设施以及
运行在其上的软件,例如图\ref{fig:cloud-arch}中展现了云计算平台的常见架构。
软件由于其便于调试、易于修改甚至代码开放(对于开源软件来说)
的特点,通常被作为传统攻击方法的切入点,攻击者往往利用恶意软件及现有软件的
安全漏洞,从而获取对被攻击系统的非授权访问,
以使被攻击系统无法提供正常服务或从被攻击系统中读取秘密信息。\cite{sw_attack}

\begin{figure}[ht]
	\centering
	\includegraphics[scale=1, page=1]{figs/figs.pdf}
	\caption{云计算平台架构}
	\label{fig:cloud-arch}
\end{figure}

而近年来,针对计算设备的底层核心硬件——处理器及其系统的攻击开始出现。
硬件系统设计复杂,并且设计者对硬件安全性的重视程度不及软件,因此容易
出现设计者难以预料到的硬件安全漏洞。并且硬件一旦制造并部署后,其安全漏洞
很难通过简单的手段进行升级修复,因此针对硬件尤其是处理器系统的安全攻击
呈现出技术含量高、破坏性大、波及范围广、难以预防及修复等特点。

RISC-V作为最新一代精简指令集,以其前所未有的开放性和可扩展性,
在计算机体系结构乃至计算机科学研究领域掀起了新的热潮。
目前业内领先的RISC-V处理器实现多为开源项目,
更为研究处理器内部运行机制的安全性提供了无与伦比的便利性。

基于RISC-V指令集开展通用处理器及其系统的安全性研究,有助于加深对硬件漏洞
产生机制的理解,从而更有针对性的提出修复方案,最终提高计算设备整体的安全性,
使得信息为人类的生产生活提供更可靠的服务。

本文将展示一种针对处理器微架构的攻击及一种针对处理器系统的攻击,
深入阐释其设计原理及实现细节,并对硬件设计者提出建议以提高硬件的安全性。


\section{国内外研究现状}{Current Status of the Research}
相比针对软件系统的攻击,针对硬件设备的攻击也可以导致相同的后果,但所受到的重视程度
远低于前者。本小结将介绍几种较为新型的,并且可以导致DoS(Denial-of-Service,拒绝服务)、非授权访问或任意代码执行
这类较为严重的后果的攻击手段,并就其实施难易程度做对比。

早在2005年,\citet{becher2005firewire}就提出了一种通过FireWire总线(也叫IEEE 1394总线)实施的
针对硬件的存储器转储攻击。FireWire是一种Apple公司推出的高速串行设备总线协议,
被广泛地应用在当时的个人计算机上,由于这种协议支持DMA(Direct Memory Access,直接存储器访问)技术,通过
FireWire总线接入计算机系统的硬件可以在无需处理器介入的情况下访问存储器,从而绕过了操作系统通常提供
的隔离机制(包括进程间和用户间隔离)。经过特殊设计的恶意FireWire硬件不仅可以直接从存储器中读取秘密信息,
甚至可以修改存储器中的内容向计算机系统植入恶意程序。但由于这种攻击需要攻击者可以物理上地访问目标
计算机系统,可以通过物理手段防御,所以计算机系统生产厂商直到2010年之后才逐渐引入IOMMU(Input–Output Memory Management Unit,输入输出存储器管理单元)
防范此类攻击的手段,如Intel公司的VT-d技术及AMD公司的AMD-Vi技术。

对于嵌入式设备的攻击,例如从智能卡中提取密钥,或从嵌入式芯片中恢复程序等,往往通过侧信道(Side Channel)来实现。
目前较为成熟的攻击方案有基于功率测量的侧信道、基于电磁场的侧信道和基于光学的侧信道等。
1999年,\citet{kocher1999differential}提出了一种名为DPA(Differential Power Analysis,差分功耗分析)
的基于功率的侧信道攻击方案,这种攻击方案通过测量微处理器在一段时间(比如在进行加解密操作时)内的瞬时功耗变化,
来推测处理器在此期间的指令流,结合加解密算法的实现,便有可能分析出用于这次加解密操作的密钥。
\citet{kocher1999differential}使用DPA作为侧信道,成功展示了从智能卡中恢复DES(Data Encryption Standard,资料加密标准,
一种对称加解密算法)密钥的过程。基于功率的侧信道攻击,需要攻击者可以物理接触硬件并可以测量
输入硬件的功率。但对于大部分嵌入式设备,如智能卡或智能家居设备,攻击者往往可以轻易获得对硬件的物理访问权限,
并且要测量设备消耗的功率并不困难,一般只需在设备的电源上串接一个小电阻,再用数字示波器采样并将数据送往计算机进行分析即可。
因此,此类攻击是嵌入式设备安全的一大主要威胁,要防范这类攻击,嵌入式设备的芯片在设计时就要采取专门的安全措施。

2014年,\citet{acoustic}提出了一种成本极低的基于声学侧信道(Acoustic Side Channel)的攻击方案,可以在较远距离(1m至10m量级)使用普通麦克风
实现对目标计算机中正在运行的加解密算法密钥的提取。这一方案利用了处理器在运算时,处理器本身及其外围电路
(如存储器电路、电源电路等)上交变的电流在PCB(Printed Circuit Board,印刷电路板)及元器件上由于压电效应
产生的声波。通过常见的音频(20Hz至20kHz)麦克风,在较远距离上仍可捕获这种声波。通过分析声波的频率及幅度,
并结合一定的统计数据,可以分析出产生声波的计算机(也就是目标计算机)正在运行的指令流,类似于基于功率的侧信道攻击,
通过对指令流结合加解密算法的分析,就有可能恢复出加解密算法使用的密钥。\citet{acoustic}展示了基于声学侧信道,使用一台普通计算机
成功恢复了另一台1m外无物理连接计算机上正在运行的GnuPG加解密软件中的RSA密钥。在文献中,作者表示若使用超声波
麦克风,可以提高攻击的效率;并且作者还展示了使用一台智能手机及其内置麦克风恢复附近一台计算机中的密钥的攻击场景。
这种基于声学侧信道的攻击方案,由于其完全采用被动观察的手段,被攻击的目标无法发现自己正在遭到攻击。并且由于可以使用
常见的麦克风及计算设备,甚至是成本更低、便携性更好的智能手机完成攻击,大大降低了实施此类攻击的难度。要避免此类攻击,
执行敏感操作的计算机的软硬件都要采取有针对性的防范措施,如采用侧信道免疫的算法并插入随机噪声以避免数据通过声学侧信道泄露。

2018年初,\citet{lipp_meltdown_2018}发现了一种广泛存在于现代高性能处理器中的硬件漏洞,可能导致运行在非特权态的
恶意用户程序读取到特权态内核的虚拟地址空间中的数据,进而通过读取映射在内核虚拟地址空间中的物理存储器段,访问整个系统
(包括其他程序甚至其他用户)的一切数据。这一漏洞被命名为Meltdown,两大主流处理器提供商,Intel和Arm均承认有部分型号受到了
这一漏洞的影响。这一漏洞利用了高性能处理器在预测执行存储器访问指令时,出于性能考虑,MMU
(Memory Management Unit,内存管理单元)并不会在地址翻译时进行特权等级检查,而是在指令提交(Commit)时才进行权限检查并
判断是否产生异常这一特点。对于恶意用户程序专门构造的针对内核地址空间的访问指令,虽然由于无法通过特权等级检查而不会被提交,从而不会对
程序员可见的状态产生影响,但由于特权等级检查的滞后性,这一指令仍然会被存储器子系统执行,可能导致缓存状态发生变化。
通过后续对缓存状态的观察,恶意用户程序就有可能推测出内核地址空间中的数据,从而实现对内核地址空间的非授权访问。
这一漏洞打破了操作系统和处理器合作提供的进程间以及用户间隔离,攻击者无需任何特权即可任意访问整个物理存储器,
甚至可以打破当前云服务中常用的容器化轻量级虚拟机(例如Docker)提供的更高级别的隔离,即从攻击者租用的虚拟机中访问
同一台宿主机上其他租户虚拟机中的内容。并且被攻击者也无法察觉到正在遭受这种攻击。这种简便性以及隐蔽性导致这一漏洞一经发现,就引起
包括个人电脑及智能手机用户、云服务提供商以及计算机制造商的一致关注,以至于Intel以牺牲性能为代价,为现有处理器提供了微码
(Microcode)补丁修复了这一漏洞。

上述四种攻击方案,均体现出了针对硬件的攻击具有技术含量高、破坏性大、波及范围广、难以预防及修复的特点。表\ref{tab:hw-attack-comp}
集中对比了这些方案的特点。

\begin{table}[!ht]
	\centering
\begin{threeparttable}[b]
\zihao{5}
\caption{常见的针对硬件的攻击方式对比}
\begin{tabular}{ccccc}
	\toprule
	攻击方式 & 攻击目标 & 需要物理接触 & 需要特制硬件 & 实施难易程度 \\
	\midrule
	FireWire DMA\cite{becher2005firewire} & 个人计算机、服务器 & 是 & 是 & 难 \\
	DPA\cite{kocher1999differential} & 嵌入式设备 & 是 & 是 & 较难 \\
	Acoustic\cite{acoustic} & 个人计算机、服务器、移动设备 & 否 & 否 & 中 \\
	Meltdown\cite{lipp_meltdown_2018} & 个人计算机、服务器、移动设备 & 否 & 否 & 易 \\
	\bottomrule
\end{tabular}
\label{tab:hw-attack-comp}
\end{threeparttable}
\end{table}


\section{研究内容及论文结构}{Research Content and Thesis Structure}
\somewords


\newpage
