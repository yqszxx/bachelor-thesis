% !TeX root = ../thesis.tex

\chapter{引言}{Introduction}


\section{研究背景及意义}{Background and Significance}
诞生于30年前的国际互联网使人类进入了信息大爆炸的时代,
尤其是近年出现的移动互联网让更多的人与设备接入了这个庞大的网络,
随之而来的是海量的信息以及对这些信息进行处理的需求。
人们部署了无数计算设备来处理信息,从我们身边的智能手表、
自动驾驶汽车,到云端的数据中心,
这些计算设备的安全性直接关系到我们的生活质量、隐私状况甚至人身安全。

用于处理数据的计算设备的架构中一般有两个主要部件:硬件基础设施以及
运行在其上的软件,例如图\ref{fig:cloud-arch}中展现了云计算平台的常见架构。
软件由于其便于调试、易于修改甚至代码开放(对于开源软件来说)
的特点,通常被作为传统攻击方法的切入点,攻击者往往利用恶意软件及现有软件的
安全漏洞,从而获取对被攻击系统的非授权访问,
以及从被攻击系统中读取秘密信息。\cite{sw_attack}

\begin{figure}[ht]
	\centering
	\includegraphics[scale=1, page=1]{figs/figs.pdf}
	\caption{云计算平台架构}
	\label{fig:cloud-arch}
\end{figure}

而近年来,针对计算设备的底层核心硬件——处理器及其系统的攻击开始出现。
硬件系统设计复杂,并且设计者对硬件安全性的重视程度不及软件,因此容易
出现设计者难以预料到的硬件安全漏洞。并且硬件一旦制造并部署后,其安全漏洞
很难通过简单的手段进行升级修复,因此针对硬件尤其是处理器系统的安全攻击
呈现出技术含量高、破坏性大、波及范围广、难以预防及修复等特点。

RISC-V作为最新一代精简指令集,以其前所未有的开放性和可扩展性,
在计算机体系结构乃至计算机科学研究领域掀起了新的热潮。
目前业内领先的RISC-V处理器实现多为开源项目,
更为研究处理器内部运行机制的安全性提供了无与伦比的便利性。

基于RISC-V指令集开展通用处理器及其系统的安全性研究,有助于加深对硬件漏洞
产生机制的理解,从而更有针对性的提出修复方案,最终提高计算设备整体的安全性,
使得信息为人类的生产生活提供更可靠的服务。

本文将展示一种针对处理器微架构的攻击及一种针对处理器系统的攻击,
深入阐释其设计原理及实现细节,并对硬件设计者提出建议以提高硬件的安全性。

\section{国内外研究现状}{Current Status of the Research}
\somewords


\section{研究内容及目标}{Research Content and Objectives}
\somewords


\section{论文组织结构}{Structure of the Thesis}
\somewords


\newpage
