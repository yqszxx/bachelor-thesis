% !TeX root = ../thesis.tex

\chapter{针对处理器微架构的攻击}{Attacks on Processor Microarchitectures}
本章将基于中科院计算所的香山高性能开源RISC-V处理器,利用处理器中的乱序执行
机制和缓存侧信道,实现一种以读取同一地址空间中存储器上任意地址内的数据为目的
的攻击方案。这种攻击方案最早在2018年初被\citet{kocher_spectre_2019}发现,
命名为Spectre(幽灵)漏洞,并在Intel Skylake及Kaby Lake微架构上得到了验证。
本研究将在香山处理器的雁栖湖微架构上实现使用与Spectre相同原理的攻击实例。


\section{高性能处理器设计理念}{High-Performance Processor Design Philosophy}
处理器设计中最主要的性能指标是单位时间内可以执行的指令数量。由于目前主流的数字
设计多采用同步设计方案(寄存器对同一个时钟边沿敏感),所以单位时间内可以执行的
指令数量就可以通过单位时间内的处理器时钟周期数量(也就是处理器的时钟频率)和每
个处理器时钟周期内可执行的指令数量(Instruction Per Cycle, IPC)相乘得到,所
以,处理器的性能与其主频和IPC都成正比,如公式\ref{eq:perf}所示。
\begin{equation}
    Performance \propto Frequency \times IPC \label{eq:perf}
\end{equation}

可见,要想提高处理器的性能,就要分别提高主频和IPC。

为了提升处理器的主频,工程师们进行了不懈的努力。处理器自上世纪70年代初进入集成
电路时代后,最初处理器的主频提升主要来自于集成电路制造工艺的优化:随着新工艺的
产生,半导体器件的漏电流及负载电容都有减小,器件间的线延迟也在减小,从而使处理
器主频不断提升:1974年,最早的个人电脑Altair 8800中使用的Intel 8080 CPU的主
频为2MHz;1992年,HP公司的PA-7100和DEC公司的Alpha 21064处理器主频突破了100MHz;
2000年,AMD公司的Athlon处理器主频达到1GHz;2002年,Intel公司的Pentium 4处理
器主频达到3GHz\cite{enwiki:clock}。

但半导体器件的动态功耗是由公式\ref{eq:pwr}决定的:
\begin{equation}
    P \propto f C V^2 \label{eq:pwr}
\end{equation}

其中$P$为功率,其主要表现形式为热能;$f$为工作频率,在处理器中即为主频;$C$为
电容,由制造工艺决定;$V$为工作电压。随着半导体制程不断下降,处理器的集成度在
升高,即单位面积内的半导体器件数量在上升,也意味着处理器的功率密度的提升。随着
越来越多的产生热量的器件集中在更小的面积内,如何控制处理器的温度以避免高温损坏
半导体器件就成了一项严峻的挑战。目前的散热技术只允许在处理器晶粒大小的面积上产
生数百瓦特的热功率,所以公式中$P$项无法有效提升。公式中的$C$项与半导体制程成
反比,但随着摩尔定律的放缓,半导体制造的特征尺寸无法快速缩减,导致电容也无法大
幅降低。而公式中的另一项$V$,受到半导体器件阈值电压的限制,目前已经降无可降。
由于上述种种限制,处理器的主频难以继续提升,这就是所谓的功耗墙。

回到公式\ref{eq:perf}中,在处理器的频率无法得到有效提高的情况下,要想提高处理
器的性能,就只能以提高处理器的IPC作为切入点。

早期的处理器设计中,每条指令都需要多个周期才能完成,例如1976年MOS Technology
公司生产的6502处理器,指令需要消耗2至7个时钟周期来完成\cite{6502manual},
每个时钟周期只有一个执行部件(如读取指令、指令译码或算术运算)会在状态机的控制下被启用。对于
这类多周期处理器,其IPC远小于1。随着计算机体系结构研究的发展,尤其是1980年之后
RISC(Reduced Instruction Set Computer,精简指令集计算机)的出现,将执行一条
指令中不同操作的执行部件以流水线的形式安排,流水级间用寄存器隔开,就可以让处于不同
执行阶段(Stage)的、正在使用不同执行部件的指令“重叠”在一起。这种流水线式的微
架构,使得处理器的IPC不断逼近1,但仍无法超过1(在一个时钟周期内执行多条指令)。
使用上述两种IPC小于1的方案设计的处理器被统称为标量处理器(Scalar Processor)。

为了追求更高的性能,就要继续想方设法提高处理器的IPC,并使之大于1,也就是处理器
需要在一个时钟周期内执行多条指令。在只有一套执行部件的处理器中,在一个时钟周期
内执行多条指令是无法实现的。但通过在流水线方案的基础上设计多套执行部件,每一个
流水级中就可以容纳与执行部件数量相当的指令。对于有$N(N = 2,3,4,\cdots)$套执行部件的处理器,每个
时钟周期最多可以提交$N$条指令,这里的$N$也被称为发射宽度。也就是在最佳情况下,使用这种方案的处理器的IPC为
$N$。使用这种IPC可以大于1方案设计的处理器被称为超标量处理器(Super-scalar Processor)。
由于这种方案在性能上的显著优势,现代高性能处理器几乎全部采用超标量设计方案。

多周期、流水线与超标量处理器中指令的执行流程如图\ref{fig:exec-cmp}所示。

\begin{figure}[ht]
	\centering
	\includegraphics[scale=1, page=2]{figs/figs.pdf}
	\caption{多周期、流水线与超标量处理器指令执行示意图}
	\label{fig:exec-cmp}
\end{figure}

对于$N$发射超标量处理器,要想使IPC尽可能接近$N$,需要在每个时钟周期向执行部件
提供合适的指令(Issue,发射),避免执行部件空转。然而在每个时钟周期都选择$N$条可以并行
执行的指令并不容易,这是因为指令间往往存在依赖关系,相互间存在依赖关系的指令无法
并行执行。指令间的依赖关系主要分为两种:
数据依赖和控制流依赖。数据依赖是指后续指令的源操作数是前序指令的目标操作数的情况,
常见于算术运算指令间或存储器访问指令与算术运算指令间。控制依赖是指分支指令后的指令
是否执行取决于分支条件判断结果的情况。

要避免数据依赖影响指令的发射,可以通过调整指令的顺序,从后续指令流中选取尽可能多
相互之间无依赖关系的指令进行发射,这种操作被称为指令的动态调度(Dynamic Scheduling)。
但调整指令执行顺序会打破指令集中规定的指令流顺序执行抽象,这就需要使用一些手段使得
指令的执行结果变得对程序员可见(指令对体系结构状态的修改,也就是提交)按照程序指令
流顺序发生,常见的此类手段为重排序缓冲区(Reorder Buffer)。这种在处理器运行过程
中动态调整程序指令顺序的技术被称为乱序执行(Out-of-Order Execution)。

而为了减小指令间的控制流依赖对指令的发射造成负面影响,可以在指令流遇到分支时预测
指令流跳转的方向以及跳转目标,这样不必等待分支条件计算,也就不会导致指令流的中断。
这种手段行之有效的原因是:分支指令的条件判断结果在统计学上存在一定的规律,并且不
同的分支指令判断结果的历史间也存在着一定的联系。目前成熟的分支预测器在绝大部分情
况下对分支跳转方向的预测正确率可达到95\%以上。此外,处理器在沿着猜测的分支方向
执行时,也会保留分支点状态的快照,这样即使分支条件的计算结果证明先前的猜测是错误的,
处理器也可以很快将状态恢复到分支点,并继续沿着正确的分支方向执行。这种在得到分支
条件计算结果前沿着预测的跳转方向继续执行的技术被称为预测式执行(Speculative Execution)。

综上所述,现代高性能处理器主要采用的设计方案是:超标量、乱序执行以及预测式执行。


\section{香山处理器南湖微架构简介}{Introduction to Nanhu Microarchitecture of Xiangshan Processor}

\subsection{乱序执行机制}{Out-of-Order Execution Mechanism}
\somewords
\subsubsection{指令重排序}
\somewords
\subsubsection{预测执行}
\somewords
\subsection{存储器子系统}{Memory Subsystem}
\somewords


\section{攻击方案}{Attack Plan}
Spectre v1 Bounds Check Bypass
Flush + Reload
\somewords


\section{可行性验证}{Feasibility Verification}
\somewords
\subsection{缓存控制与检测}{Cache Manipulation and Measurement}
flush cache line
Timed read
\subsection{误导预测执行流}{Misleading Speculative Execution Stream}
Training
\subsection{香山南湖架构攻击验证}{Attack Verification on Nanhu Microarchitecture}
\somewords


\section{本章小结}{Chapter Summary}
\somewords


\newpage
