% !TEX program = xelatex
\documentclass[degree=undergraduate,bibstyle=numerical,font=empty]{xmuthesis}

\usepackage{geometry}
\geometry{a4paper,scale=0.8}

\DeclareMathOperator*{\argmin}{argmin}

\usepackage{paralist}

\usepackage{rotating}

\makeatletter
\newenvironment{breakablealgorithm}
  {% \begin{breakablealgorithm}
   \begin{center}
     \refstepcounter{algorithm}% New algorithm
     \hrule height.8pt depth0pt \kern2pt% \@fs@pre for \@fs@ruled
     \renewcommand{\caption}[2][\relax]{% Make a new \caption
       {\raggedright\textbf{\fname@algorithm~\thealgorithm} ##2\par}%
       \ifx\relax##1\relax % #1 is \relax
         \addcontentsline{loa}{algorithm}{\protect\numberline{\thealgorithm}##2}%
       \else % #1 is not \relax
         \addcontentsline{loa}{algorithm}{\protect\numberline{\thealgorithm}##1}%
       \fi
       \kern2pt\hrule\kern2pt
     }
  }{% \end{breakablealgorithm}
     \kern2pt\hrule\relax% \@fs@post for \@fs@ruled
   \end{center}
  }
\makeatother

\usepackage{array}
\newcolumntype{x}[1]{>{\centering\arraybackslash\hspace{0pt}}p{#1}}

\xmusetup{
    author                  = 韩博阳 ,
    title                   = 基于 RISC-V 指令集的通用处理器安全性研究 ,
    date                    = \today , % 二〇一九年二月二十八日
    class                   = 2018级 ,
    studentnumber           = 35320182200138 ,
    department              = 电子科学与技术学院 ,
    major                   = 集成电路设计与集成系统 ,
    advisor                 = 周剑扬 \quad 教授 ,
    otheradvisor            = 无 ,
    team                    =  ,
    fundteam                =  ,
    degree                  = 本\quad 科 ,
    englishtitle            = Research on General-Purpose Processor Security Based on RISC-V Instruction Set ,
    majorordouble           = 主修 , % 辅修
    lab                     =  ,
    % 以下几项本科生无需填写,也不用删除
    classified_code         =  1234                       , % 分类号
    security_classification =  公开                       , % 内部 秘密 机密 公开 等
    UDC                     =  5                          , % 国际十进分类法
    submit_date             =  2020 年 8 月 10 日         , % 论文提交日期
    defense_date            =  2020 年 8 月 11 日         , % 论文答辩日期
    conferred_date          =  2020 年 8 月 21 日         , % 学位授予日期
    chairman                =  张三                       , % 答辩委员会主席
    referee                 =  李四                       , % 评阅人
}
\usepackage{xmulogo}
\listfiles
\begin{document}
\maketitle
% !TeX root = ../thesis.tex

\chapter*{致谢}
受到疫情影响,很遗憾不能在本科生活即将结束的时候,站在厦门大学的校园中再次欣赏
盛放的凤凰花,但我不会因为时间与距离而淡忘四年间给予我帮助的每一个人,让我把对
他们由衷的感谢记入这篇我的本科时期的收官之作中。

首先我要感谢我的导师周剑扬教授。感谢周老师在我的本科四年期间对我全方位的指导,
不仅在技术问题上他会深入浅出地给我讲解、与我讨论,在学术研究的其他方面,比如项目
申请、物资采购及进度管理等细节上,他也事无巨细地将经验传授给我。还要感谢周老师
邀请我加入他所带领的数字设计实验室,让我能有机会参与到研究生学长学姐的项目中,
极大地提高了我的工程能力,并且让我能使用实验室先进的硬件设备,这些设备是我学习
和研究的得力助手。但更重要的是,如果不是他在2018年12月24日晚上作为我学习计算
机体系结构的启蒙者,向我介绍了正在蓬勃发展的RISC-V指令集,
我很可能会错过这个让我感到兴奋的领域,也一定无法达成今天的成就。

我还要感谢中国科学院软件研究所的吴伟老师、中国科学院计算技术研究所的包云岗教授和
鹏城实验室的解壁伟博士,他们为我提供了无与伦比的实习研究环境,让我有机会接触领域
内最前沿的技术、参与具有国际影响力的会议,在他们的团队中工作是我的荣幸。

最后,我诚挚地将本文献给我的家人,他们是我一路向前的动力源泉和坚强后盾。
% !TeX root = ../thesis.tex

% 400字
\begin{abstract}*
    中文摘要
    \keywords*{关键词1;关键词2;关键词3}
\end{abstract}
\begin{abstract}
    English abstract. 
    \keywords{keyword1; keyword2; keyword3}
\end{abstract}

\xmutableofcontents % 双语目录
% !TeX root = ../thesis.tex

\chapter{引言}{Introduction}


\section{研究背景及意义}{Background and Significance}
诞生于30年前的国际互联网使人类进入了信息大爆炸的时代,
尤其是近年出现的移动互联网让更多的人与设备接入了这个庞大的网络,
随之而来的是海量的信息以及对这些信息进行处理的需求。
人们部署了无数计算设备来处理信息,从我们身边的智能手表、
自动驾驶汽车,到云端的数据中心,
这些计算设备的安全性直接关系到我们的生活质量、隐私状况甚至人身安全。

用于处理数据的计算设备的架构中一般有两个主要部件:硬件基础设施以及
运行在其上的软件,例如图\ref{fig:cloud-arch}中展现了云计算平台的常见架构。
软件由于其便于调试、易于修改甚至代码开放(对于开源软件来说)
的特点,通常被作为传统攻击方法的切入点,攻击者往往利用恶意软件及现有软件的
安全漏洞,从而获取对被攻击系统的非授权访问,
以及从被攻击系统中读取秘密信息。\cite{sw_attack}

\begin{figure}[ht]
	\centering
	\includegraphics[scale=1, page=1]{figs/figs.pdf}
	\caption{云计算平台架构}
	\label{fig:cloud-arch}
\end{figure}

而近年来,针对计算设备的底层核心硬件——处理器及其系统的攻击开始出现。
硬件系统设计复杂,并且设计者对硬件安全性的重视程度不及软件,因此容易
出现设计者难以预料到的硬件安全漏洞。并且硬件一旦制造并部署后,其安全漏洞
很难通过简单的手段进行升级修复,因此针对硬件尤其是处理器系统的安全攻击
呈现出技术含量高、破坏性大、波及范围广、难以预防及修复等特点。

RISC-V作为最新一代精简指令集,以其前所未有的开放性和可扩展性,
在计算机体系结构乃至计算机科学研究领域掀起了新的热潮。
目前业内领先的RISC-V处理器实现多为开源项目,
更为研究处理器内部运行机制的安全性提供了无与伦比的便利性。

基于RISC-V指令集开展通用处理器及其系统的安全性研究,有助于加深对硬件漏洞
产生机制的理解,从而更有针对性的提出修复方案,最终提高计算设备整体的安全性,
使得信息为人类的生产生活提供更可靠的服务。

本文将展示一种针对处理器微架构的攻击及一种针对处理器系统的攻击,
深入阐释其设计原理及实现细节,并对硬件设计者提出建议以提高硬件的安全性。

\section{国内外研究现状}{Current Status of the Research}
\somewords


\section{研究内容及目标}{Research Content and Objectives}
\somewords


\section{论文组织结构}{Structure of the Thesis}
\somewords


\newpage

% !TeX root = ../thesis.tex

\chapter{针对处理器微架构的攻击}{Attacks on Processor Microarchitectures}


\section{高性能处理器设计理念}{High-Performance Processor Design Philosophy}
\somewords


\section{香山处理器南湖微架构简介}{Introduction to Nanhu Microarchitecture of Xiangshan Processor}
\somewords
\subsection{乱序执行机制}{Out-of-Order Execution Mechanism}
\somewords
\subsubsection{指令重排序}
\somewords
\subsubsection{预测执行}
\somewords
\subsection{存储器子系统}{Memory Subsystem}
\somewords


\section{攻击方案}{Attack Plan}
Spectre v1 Bounds Check Bypass
Flush + Reload
\somewords


\section{可行性验证}{Feasibility Verification}
\somewords
\subsection{缓存控制与检测}{Cache Manipulation and Measurement}
flush cache line
Timed read
\subsection{误导预测执行流}{Misleading Speculative Execution Stream}
Training
\subsection{香山南湖架构攻击验证}{Attack Verification on Nanhu Microarchitecture}
\somewords


\section{本章小结}{Chapter Summary}
\somewords


\newpage

% !TeX root = ../thesis.tex

\chapter{针对处理器系统的攻击}{Attacks on Processor Systems}


\section{处理器系统及其通常结构}{Processor Systems and Their Common Structures}
\somewords


\section{SiFive Freedom U740 SoC结构}{Structure of the SiFive Freedom U740 SoC}
\somewords
\subsection{片上互联}{On-Chip Interconnect}
\somewords
\subsection{PCIe 子系统}{PCIe Subsystem}
\somewords


\section{基于DMA的攻击}{DMA-Based Attack}
\somewords


\section{PCIe DMA攻击实例}{An Attack Example Based on PCIe DMA}
\somewords
\subsection{PCIe Endpoint实现}{Implementation of PCIe Endpoint}
\somewords
\subsection{Linux 物理地址获取}{Acquisition of Physical Address on Linux}
\somewords
\subsection{秘密信息读取}{Readout of Secret Data}
\somewords

\section{本章小结}{Chapter Summary}
\somewords


\newpage

% !TeX root = ../thesis.tex

\chapter{总结与展望}{Conclusion and Outlook}



\newpage

\nocite{*} 
\bibliography{ref}
\appendix
\chapter{Spectre漏洞利用代码}{Spectre Proof-of-Concept Code} \label{app:spectre-code}
\lstinputlisting[language=c]{body/spectre.c}

\chapter{虚拟到物理内存地址翻译程序}{Virtual to Physical Address Translator} \label{app:v2p}
\lstinputlisting[language=c]{body/v2p.c}

\chapter{PCIe设计框图}{Block Design of PCIe Endpoint} \label{app:pcie-ep}
\begin{sidewaysfigure}
	\centering
	\includegraphics[width=\textwidth]{figs/pcie-ep.pdf}
	\caption{软件可控PCIe Endpoint Block Design}
\end{sidewaysfigure}
% \backmatter
% 后记
\end{document}
